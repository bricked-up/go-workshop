% Created 2025-02-11 Tue 15:30
% Intended LaTeX compiler: pdflatex
\documentclass[presentation]{beamer}
\usepackage[utf8]{inputenc}
\usepackage[T1]{fontenc}
\usepackage{graphicx}
\usepackage{longtable}
\usepackage{wrapfig}
\usepackage{rotating}
\usepackage[normalem]{ulem}
\usepackage{amsmath}
\usepackage{amssymb}
\usepackage{capt-of}
\usepackage{hyperref}
\usetheme{default}
\author{Rumen Mitov}
\date{\today}
\title{Go Workshop}
\hypersetup{
 pdfauthor={Rumen Mitov},
 pdftitle={Go Workshop},
 pdfkeywords={},
 pdfsubject={},
 pdfcreator={Emacs 29.4 (Org mode 9.6.15)}, 
 pdflang={English}}
\begin{document}

\maketitle
\begin{frame}{Outline}
\tableofcontents
\end{frame}


\begin{frame}[label={sec:org5c24fbe}]{Go: A Quick Introduction}
\begin{itemize}
\item Systems programming language
\item Garbage-collected
\item Large standard library
\item Great developer tooling
\end{itemize}
\end{frame}

\begin{frame}[label={sec:org790e5c1}]{Installation}
\href{https://go.dev/doc/install}{Golang Installation}
\end{frame}

\begin{frame}[label={sec:orgbc6e130}]{Tour of Go}
\href{https://go.dev/tour/list}{Start the tour!}
\end{frame}

\begin{frame}[label={sec:org3742f35},fragile]{Workshop: Simple Go Endpoint}
 Run from the root directory of this repo: \texttt{go run ./src}
\end{frame}

\begin{frame}[label={sec:orgf891335}]{Further Web Examples}
\href{https://gowebexamples.com/}{Go web examples}
\end{frame}

\begin{frame}[label={sec:orgd8c5bfd}]{Organization of our Repo}
\end{frame}

\begin{frame}[label={sec:org9508b8e}]{Directory Structure}
\begin{itemize}
\item \alert{src} - for main endpoint handling / routing
\item \alert{sql} - sql scripts to initiate database tables / populate database with dummy data
\item \alert{utils} - common utility functions
\item \alert{assets} - media (videos, images, etc.)
\item \alert{docs} - documentation
\end{itemize}
\end{frame}

\begin{frame}[label={sec:orga819eb8},fragile]{Branch Policy}
 \begin{itemize}
\item each new feature / fix is its own branch
\item commit freely to that branch
\item feature must work in order to be merged with the \texttt{dev} branch
\item developers submit pull requests to \texttt{dev} branch (\hyperlink{sec:orgf69369e}{meaningful} commit message)
\item backend lead is responsible for making sure \texttt{dev} works before merging with \texttt{master}
\item \alert{NOTE}: Only \texttt{dev} merges with \texttt{master}
\end{itemize}
\end{frame}

\begin{frame}[label={sec:orgf69369e},fragile]{Commit Message Prefix}
 This applies to final commits (i.e. commits when merging):
\begin{itemize}
\item \alert{feat(<feature>)}: description
\item \alert{refactor(<feature>)}: description
\item \alert{fix(<feature>):} description
\item \alert{hotfix(<bug>)}: description
\item \alert{bug(<bug>)}: description
\item \alert{tests(<feature>)}: description
\end{itemize}


\begin{verbatim}
feat(login): created login endpoint
\end{verbatim}
\end{frame}

\begin{frame}[label={sec:org4b7e846}]{Merging}
\begin{itemize}
\item \alert{Squash merge} if your feature is ready to be merged upstream!
\item \alert{Rebase} if you are still working on your code, but you need to pull the latest changes!
\end{itemize}
\end{frame}
\end{document}
